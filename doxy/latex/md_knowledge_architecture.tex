{\bfseries Preview\+: 基于 Reactor 的半同步/半异步并发模式 + 同步 epoll (ET) 事件循环 + 多线程 (线程池)} 



\paragraph*{整体结构与详细工作流}

 



\subsubsection*{各大模块}

\paragraph*{1. Epoll 事件循环}

{\bfseries \hyperlink{pink__epoll_8h}{pink\+\_\+epoll.\+h}$\vert$cpp + \hyperlink{pink__epoll__tool_8h}{tools/pink\+\_\+epoll\+\_\+tool.\+h}$\vert$cpp}

在整体架构中,epoll 属于同步的部分,由主线程同步运行。


\begin{DoxyItemize}
\item {\bfseries 结构}
\end{DoxyItemize}
\begin{DoxyEnumerate}
\item 采用 ET 模式。(同时实现了 LT 模式。)
\item 注册的事件类型\+: (1)新连接请求到来(2)连接socket上读就绪(3)连接socket上写就绪
\item 只负责接受连接,分发读取/写出任务,以及定时器超时任务。
\item 定期处理超时事件\+: 目前每5秒。
\end{DoxyEnumerate}

\paragraph*{2. 连接池}

{\bfseries \hyperlink{pink__conn__pool_8h}{pink\+\_\+conn\+\_\+pool.\+h}}

为了节约连接体申请内存的时间开销,采用连接池进行优化。


\begin{DoxyItemize}
\item {\bfseries 结构}
\end{DoxyItemize}
\begin{DoxyEnumerate}
\item 通过预先分配 pre\+\_\+conn\+\_\+number 个连接体来构建连接池。
\item 核心\+: list$<$\+T$\ast$$>$ conn\+\_\+lst 连接体指针链表。
\item 每次从表头获取连接体指针(即原先的 new),每次归还连接体指针到表尾(即原先的 delete)。
\end{DoxyEnumerate}
\begin{DoxyItemize}
\item {\bfseries 题外话}
\end{DoxyItemize}
\begin{DoxyEnumerate}
\item 一开始实现连接池时,打算用连接体数组,并搭配最小堆来存放连接体下标,每次分配最小可用下标的连接体 O(log\+K)。分配连接的时间效率肯定不如现在的链表形式 O(1)。但是最小堆的内存局部性更强。
\item 实际上直接用连接体数组,并用 fd 来直接做下标也是可行的。但总觉得不够优雅。
\end{DoxyEnumerate}

\paragraph*{3. 时间堆}

{\bfseries \hyperlink{pink__conn__timer_8h}{pink\+\_\+conn\+\_\+timer.\+h} + \hyperlink{pink__epoll_8h}{pink\+\_\+epoll.\+h}$\vert$cpp}


\begin{DoxyItemize}
\item {\bfseries 结构}
\end{DoxyItemize}
\begin{DoxyEnumerate}
\item 采用自实现的最小堆,存放定时器的指针。
\item 每次处理时不断从顶部取出定时器,并判断是否超时。
\item 删除定时器采用懒惰删除,给定时器打上 cancled 标签。
\end{DoxyEnumerate}

\paragraph*{4. 线程池}

{\bfseries \hyperlink{pink__thread__pool_8h}{pink\+\_\+thread\+\_\+pool.\+h}}


\begin{DoxyItemize}
\item {\bfseries 结构}
\end{DoxyItemize}
\begin{DoxyEnumerate}
\item {\bfseries 半同步/半反应堆模式$\ast$$\ast$,工作队列($\ast$$\ast$同步})负责任务获取与分发,工作线程({\bfseries 异步})负责处理任务。
\item 工作队列\+: list$<$pair$<$\+T$\ast$, int$>$$>$,每个任务包含结构体和一个标签 flag,用于标记 R\+E\+A\+D/\+W\+R\+I\+TE 任务。
\item 工作线程\+: shared\+\_\+ptr$<$pthread\+\_\+t$>$ threads,线程线程标识符数组,预分配大小。
\end{DoxyEnumerate}
\begin{DoxyItemize}
\item {\bfseries 优点}
\end{DoxyItemize}
\begin{DoxyEnumerate}
\item 以空间换取时间效率。
\item 线程池适用于高并发多线程,但每个线程运行时间较短的情况。
\item 通用性高,比较 general,采用模板元编程。
\end{DoxyEnumerate}
\begin{DoxyItemize}
\item {\bfseries 限制}
\end{DoxyItemize}
\begin{DoxyEnumerate}
\item 要求客户请求都是无状态的,同一个连接上的不同请求可能由不同的线程处理。
\item 容易产生惊群效应,需要特殊处理。
\end{DoxyEnumerate}

\paragraph*{5. 工作线程}


\begin{DoxyEnumerate}
\item 工作线程取出读取/写出任务后,回调 H\+T\+TP 连接类的 process() 函数。
\item 回调一次 process() 函数负责\+:
\end{DoxyEnumerate}

(1)从内核缓冲区读取数据到用户缓冲区\+: 非阻塞 read,(遇到 E\+A\+G\+A\+IN 则返回)

(2)处理 H\+T\+TP 请求报文

(3)构建 H\+T\+TP 响应报文

(4)把响应报文写入到内核缓冲区\+: 非阻塞 write,(遇到 E\+A\+G\+A\+IN 则返回)


\begin{DoxyEnumerate}
\item {\bfseries 如果能一次性处理完则不再通知 epoll} 
\end{DoxyEnumerate}